\documentclass[11pt, rgb]{scrbook}

%%%%%%%%%%%%%%%%%%%%%%%%%%%%%%%%
% Paket Theme Konstanz         %
%%%%%%%%%%%%%%%%%%%%%%%%%%%%%%%%

% Alle verwendeten Makros und Umgebeungen, die zur Erstellung von
% PDF-Dokumenten und Abschlussarbeiten benötigt werden, können
% aus dem Paket themeKonstanz geladen werden, welches sich in der
% Datei themeKonstanz.sty befindet.
%
% Möchte man dieses Dokument mit XeLaTeX anstatt PDFLatex kompilieren,
% um die Systemschrift Arial zu verwenden, dann muss zusätzlich das
% Add-On Paket themeKonstanzXelatexAddOn, welches sich in der Datei
% themeKonstanzXelatexAddOn.sty befindet VOR dem eigentlichen
% Paket themeKonstanz geladen werden (siehe unten).
%
% Möchte man ein Style Add-On verwenden, das aus meiner
% Bachelorarbeit aus dem Jahr 2015 stammt, so muss NACH dem Laden
% des Paketes themeKonstanz dieses zusätzliche Paket
% themeKonstanzStyleAddOn, welches sich in der Datei
% themeKonstanzStyleAddOn.sty befindet, geladen werden. Dabei 
% werden die Überschriften der Kapitel, der Abschnitte, der Unter-
% Abschnitte und des Unterunterabschnittes geändert. Dieses
% Add-On wird vor allem für Abschlussarbeiten empfohlen, die
% nicht zu farbig sein sollten und nicht so strikt an das
% Corporate Design der Universität Konstanz gebunden sind.

%\usepackage{themeKonstanzXelatexAddOn} % XeLaTeX mit Schriftart Arial, VOR dem Standardpaket importieren
\usepackage{themeKonstanz} % Muss immer verwendet werden (Standardpaket)
%\usepackage{themeKonstanzStyleAddOn} % Style Add-On für andere Überschriften, NACH dem Standardpaket importieren
\usepackage[square,sort,comma,numbers]{natbib}
\usepackage{import}

%%%%%%%%%%%%%%%%%%%%%%%%%%%%%%%%
% Papierformat                 %
%%%%%%%%%%%%%%%%%%%%%%%%%%%%%%%%

% Mit Hilfe des Makros
%
%    \format{<key>}
%
% kann das Papierformat angepasst werden.
%
% Zur Auswahl stehen folgende Papierformate:
%
%   Schlüssel  Beschreibung      Höhe      Breite
%   -----------------------------------------------
%   a3         DinA3 Hochformat  42.0  cm  29.7  cm
%   a3quer     DinA3 Querformat  29.7  cm  42.0  cm
%   a4         DinA4 Hochformat  29.7  cm  21.0  cm
%   a4quer     DinA4 Querformat  21.0  cm  29.7  cm
%   a5         DinA5 Hochformat  21.0  cm  14.8  cm
%   a5quer     DinA5 Querformat  14.8  cm  21.0  cm
%
% Falls keiner der unterstützten Schlüssel übergeben wird,
% ist DinA4 Hochformat der Standard.

% Wähle hier DinA4 Hochformat mit dem Schlüssel a4 aus.

\format{a4}



%%%%%%%%%%%%%%%%%%%%%%%%%%%%%%%%
% Dokumentinformationen        %
%%%%%%%%%%%%%%%%%%%%%%%%%%%%%%%%

% Für das Dokument können mit Hilfe der Makros
%     
%    \date{...}
%    \year{...}
%    \author{...}
%    \title{...}
%    \subtitle{...}
%    \unisection{...}
%    \department{...}
%    \supervisorOne{...}
%    \supervisorTwo{...}
%
% das Datum, das Jahr, der Autor / die Autoren, der Titel, der
% Unteritel des Dokumentes, die Sektion an der Universität,
% der Fachbereich an der Universität, der Erstgutachter und der
% Zweitgutachter bestimmt werden.
%
% Je nachdem, ob die Corporate Design Titelseite oder die
% Thesis-Titelseite ausgewählt wird, werden die entsprechenden
% Angaben dargestellt und verwendet. Außerdem benutzen die 
% Kopf- und Fusszeile diese Informationen (siehe weiter unten).

\year{2017}
\author{\\ Dennis Rosskothen}
\title{Influence of \textit{TG probiotic} as a probiotic feed supplement on rainbow trout (\textit{Oncorhynchus mykiss})}
\unisection{Biology}
\department{Biological Siences}
\supervisorOne{Dr. Jasminca Behrman-Godel}
\supervisorTwo{Dr. Anselm Rink}



%%%%%%%%%%%%%%%%%%%%%%%%%%%%%%%%
% Kopf und Fusszeile           %
%%%%%%%%%%%%%%%%%%%%%%%%%%%%%%%%

% Mit Hilfe des Makros
%
%    \headFoot
%
% kann aus 20 vordefinierten Kopf- und Fusszeilen
% gewählt werden.
%
% Dieses Makro besitzt ein Argument, nämlich die
% ID aus den 20 vordefinierten Kopf- und Fusszeilen:
%
%    \headFoot{<ID>}
%
% Bei den IDs 1, 2, 5, 6, 9, 10, 13, 14, 17 und 18 
% mit | wird auf ungerade und gerade Seite geachtet.
%
% Bei den restlichen IDs ist die Kopf- und Fusszeile auf 
% ungeraden und geraden Seiten gleich.
%
% Dies soll vor allem dem Benutzer die Entscheidung
% überlassen, ob er das Dokument einseitig oder doppelseitig
% ausdruckt.
%
%
%   ID    Beschreibung
%   ------------------
%   1     Kopfzeile: Seite --- Kapitel --- Autor | Autor --- Kapitel --- Seite
%         Fusszeile:
%
%   2     Kopfzeile: 
%         Fusszeile: Seite --- Kapitel --- Autor | Autor --- Kapitel --- Seite
%
%   3     Kopfzeile: Autor --- Kapitel --- Seite
%         Fusszeile:
%
%   4     Kopfzeile: 
%         Fusszeile: Autor --- Kapitel --- Seite
%
%   5     Kopfzeile: Seite --- Autor --- Kapitel | Kapitel --- Autor --- Seite
%         Fusszeile:
%
%   6     Kopfzeile: 
%         Fusszeile: Seite --- Autor --- Kapitel | Kapitel --- Autor --- Seite
%
%   7     Kopfzeile: Kapitel --- Autor --- Seite
%         Fusszeile:
%
%   8     Kopfzeile: 
%         Fusszeile: Kapitel --- Autor --- Seite
%
%   9     Kopfzeile: Seite --- --- Kapitel | Name --- --- Seite
%         Fusszeile:
%
%   10    Kopfzeile: 
%         Fusszeile: Seite --- --- Kapitel | Name --- --- Seite
%
%   11    Kopfzeile: Kapitel --- --- Seite
%         Fusszeile:
%
%   12    Kopfzeile: 
%         Fusszeile: Kapitel --- --- Seite
%
%   13    Kopfzeile: Seite --- --- Autor | Kapitel --- --- Seite
%         Fusszeile:
%
%   14    Kopfzeile:
%         Fusszeile: Seite --- --- Autor | Kapitel --- --- Seite
%
%   15    Kopfzeile: Autor --- --- Seite
%         Fusszeile:
%
%   16    Kopfzeile:
%         Fusszeile: Autor --- --- Seite
%
%   17    Kopfzeile: Seite --- --- | --- --- Seite
%         Fusszeile:
%
%   18    Kopfzeile: 
%         Fusszeile: Seite --- --- | --- --- Seite
%
%   19    Kopfzeile: --- --- Seite
%         Fusszeile:
%
%   20    Kopfzeile: 
%         Fusszeile: --- --- Seite
%
% Weitere individuelle Kopf- und Fusszeilen können in der .sty Datei angepasst werden.

% Wähle die ID 14 mit 
%    Kopfzeile:
%    Fusszeile: Seite --- --- Autor | Kapitel --- --- Seite

\headFoot{17}



%%%%%%%%%%%%%%%%%%%%%%%%%%%%%%%%%%%%%%%%%%%%%%%%%%%%%%%%%
% Begin vom Dokument                                    %
%%%%%%%%%%%%%%%%%%%%%%%%%%%%%%%%%%%%%%%%%%%%%%%%%%%%%%%%%

\begin{document}

%%%%%%%%%%%%%%%%%%%%%%%%%%%%%%%%%%%%%%%%%%%%%%%%%%%%%%%%%
% Titelseite                                 %
%%%%%%%%%%%%%%%%%%%%%%%%%%%%%%%%%%%%%%%%%%%%%%%%%%%%%%%%%


\thesistitlepage[language=english]{Master Thesis}
\cleardoublepage



%_________ frontmatter (und mainmatter) wurden nur via scrBOOK möglich gemacht. frontmatter sind die vorangestellten Seiten die nicht im Inhaltsverzeichnis erscheinen
\frontmatter


\pagenumbering{roman}
%
%
% File: abstract.tex
% Author: V?ctor Bre?a-Medina
% Description: Contains the text for thesis abstract
%
% UoB guidelines:
%
% Each copy must include an abstract or summary of the dissertation in not
% more than 300 words, on one side of A4, which should be single-spaced in a
% font size in the range 10 to 12. If the dissertation is in a language other
% than English, an abstract in that language and an abstract in English must
% be included.

\chapter*{Abstract}

Lorem ipsum psum dolor sit amet, consetetur sadipscing elitr, sed diam nonumy eirmod tempor invidunt ut labore et dolore magna aliquyam erat, sed diam voluptua. At vero eos et accusam et justo duo dolores et ea rebum. Stet clita kasd gubergren, no sea takimata sanctus est Lorem ipsum dolor sit amet. Lorem ipsum dolor sit amet, consetetur sadipscing elitr, sed diam nonumy eirmod tempor invidunt ut labore et dolore magna aliquyam erat, sed diam voluptua. At vero eos et accusam et justo duo dolores et ea rebum. Stet clita kasd gubergren, no sea takimata sanctus est Lorem ipsum dolor sit amet.\\
\\
Der Abstract wird in deutscher oder englischer Sprache verfasst, englisch wenn die Arbeit im Deutschen verfasst wurde, deutsch wenn die Arbeit im englischen verfasst wurde. Der Text wird erst dann verfasst, wenn Ergebnissteil und Diskussion im Kern geschrieben sind.



%\end{SingleSpace}
\clearpage
\cleardoublepage
%

% file: dedication.tex
% author: V?ctor Bre?a-Medina
% description: Contains the text for thesis dedication
%

\chapter*{Dedication and acknowledgements}
%
This thesis is dedicated to my mentor and father Klaus Rosskothen. I am sure you watch all my steps closely and i hope that i can make you proud in all the decisions and steps that i took and the ones that lay ahead of me. You were and always will be my biggest motivational driver. Rest in peace dad.
\\\\
I also want to aknowledge all the efforts that the company 20Green shouldered to make this collaboration possible.

%
\clearpage
\cleardoublepage
%
%
% File: declaration.tex
% Author: V?ctor Bre?a-Medina
% Description: Contains the declaration page
%
% UoB guidelines:
%
% Author's declaration
%
% I declare that the work in this dissertation was carried out in accordance
% with the requirements of the University's Regulations and Code of Practice
% for Research Degree Programmes and that it has not been submitted for any
% other academic award. Except where indicated by specific reference in the
% text, the work is the candidate's own work. Work done in collaboration with,
% or with the assistance of, others, is indicated as such. Any views expressed
% in the dissertation are those of the author.
%
% SIGNED: .............................................................
% DATE:..........................
%


\chapter*{Author's declaration}

\begin{quote}
I declare that the work in this dissertation was carried out in accordance with the requirements of  the University's Regulations and Code of Practice for Research Degree Programmes and that it  has not been submitted for any other academic award. Except where indicated by specific  reference in the text, the work is the candidate's own work. Work done in collaboration with, or with the assistance of, others, is indicated as such. Any views expressed in the dissertation are those of the author.

\vspace{1.5cm}
\noindent
\hspace{-0.75cm}\textsc{SIGNED: .................................................... DATE: ..........................................}
\end{quote}

\clearpage
\cleardoublepage


%_____________ mainmatter macht die Nummerierung wieder von römisch nach normal

\mainmatter

%%%%%%%%%%%%%%%%%%%%%%%%%%%%%%%%
% Inhaltsverzeichnis           %
%%%%%%%%%%%%%%%%%%%%%%%%%%%%%%%%

% Das Inhaltsverzeichnis kann wie gewohnt mit dem Makro
%
    \tableofcontents
%
% erstellt werden.

%\tableofcontents



%%%%%%%%%%%%%%%%%%%%%%%%%%%%%%%%
% Abbildungsverzeichnis        %
%%%%%%%%%%%%%%%%%%%%%%%%%%%%%%%%

% Das Abbildungsverzeichnis kann wie gewohnt mit dem Makro
%
%    \listoffigures
%
% erstellt werden.
%
% Möchte man das Abbildungsverzeichnis zum Inhaltsverzeichnis
% hinzufügen, so kann dies mit dem Makro
% 
%    \addcontentsline{toc}{chapter}{Abbildungsverzeichnis}
%
% hinzugefügt werden.

%\listoffigures
%\addcontentsline{toc}{chapter}{Abbildungsverzeichnis}



%%%%%%%%%%%%%%%%%%%%%%%%%%%%%%%%
% Tabellenverzeichnis          %
%%%%%%%%%%%%%%%%%%%%%%%%%%%%%%%%

% Das Tabellenverzeichnis kann wie gewohnt mit dem Makro
%
%  \listoftables
%
% erstellt werden.
%
% Möchte man das Tabellenverzeichnis zum Inhaltsverzeichnis
% hinzufügen, so kann dies mit dem Makro
% 
%    \addcontentsline{toc}{chapter}{Tabellenverzeichnis}
%
% hinzugefügt werden.

%\listoftables
%\addcontentsline{toc}{chapter}{Tabellenverzeichnis}




% \rmfamily % Auskommentieren für eine Schrift mit serifen / Kommentieren für eine serifenlose Schrift
\normalsize


%_________input von ausgelagerten chapters



\chapter{Introduction}
\section{"of fish and men"}
Global Aquaculture is often referred to as "The Blue Revolution", a remarkable solution for the worlds craving hunger of easy available and healthy animal protein. Since the 1980s fish populations all over the world start becoming more and more overexploited and/or collapsed, with an annual 10PROZENT in 1974 up to a concerning 29,9PROZENT in the year 2009 (Citation worldoceanreview). While 50PROZENT of all fish populations are regularly fully exploited, only a mere 10 of all fish populations is moderately used(same citation). Around the year 1995 landings by global capture fisheries
have leveled off at around 80-90 mio.t. per year, mostly owing to the issue of economical overfishing, rather than ecological overfishing(citation needed). 

While overfishing in general lead to devastating and catastrophic situations in ecosystems all over the world, changing them drastically sometimes even eradicating them entirely, the global demand for seafood still rose.


\section{Theory and Hypothesis}
The following section offers some tentative thoughts on the issue. As the question has, to the best of my knowledge, not explicitly been tackled by academic literature, these thoughts are largely of exploratory nature and necessitate further development. Nevertheless, I formulate two competing hypotheses that plausibly explain the relationship. In this outline, I gloss over definitions and operationalizations of (social) trust and regime type.

One may suppose that autocratic regimes actively attempt to destroy generalized trust as it could aid in alleviating collective action problems possible opposition forces face (Magaloni 2010)\nocite{magaloni2010game}: If individuals generally trust their fellow citizens, they may be more inclined to engage in anti-state behavior of some degree. To keep power and preempt opposition, the regime could foster citizen spy systems (e.g. the Stasi in the GDR), forbid social networks (e.g. Turkey under Erdogan) and restrict the right of assembly. Thus, one could easily suppose:

\begin{center}
H1a: Individuals in autocratic regimes are less likely to generally trust others
\end{center}

Just as easily, one can imagine the regime actively fostering trust in certain groups. For example, a regime that favors one ethnicity over another would try to strengthen that groups social cohesion. This corresponds to Putnam's idea of bonding social capital. Moreover, generalized trust may instill a patriotic feeling of loyalty towards the regime (e.g. in a monarchy). Then, the competing hypothesis would be:

\begin{center}
H1b: Individuals in autocratic regimes are more likely to generally trust others
\end{center}

Finally, there is one caveat: Citizens that favor (and/or benefit from) the regime, may display more generalized trust. I am going to account for this using interaction terms (e.g. perception of economic situation $x$ autocratic regime).

\section{Data}
Individual-level survey data will be taken from the WVS-Survey – possibly in longitudinal form. Country-level data is available with the World Bank. Classification into autocratic or democratic regimes will be taken from Geddes et al (2014)\nocite{geddes2014autocratic}.One may suppose that autocratic regimes actively attempt to destroy generalized trust as it could aid in alleviating collective action problems possible opposition forces face (Magaloni 2010)\nocite{magaloni2010game}: If individuals generally trust their fellow citizens, they may be more inclined to engage in anti-state behavior of some degree. To keep power and preempt opposition, the regime could foster citizen spy systems (e.g. the Stasi in the GDR), forbid social networks (e.g. Turkey under Erdogan) and restrict the right of assembly. Thus, one could easily suppose:

\section{Method and Analysis}

As explained above, the proposed model would look as follows:

\begin{center}

$y_i=\alpha+\beta RegimeType_c + X^c + X^i + \epsilon$,

\end{center}

Where $y_i$ is the outcome variable of interest on the individual level: (level of) social trust, $RegimeType_c$ is the independent variable at the country level, $X^c$ is a vector of country-level control variables (e.g. corruption, log of GDP, log of population size), $X^i$ is a vector of individual-level control variables (demographic controls).

As is common in the field of trust research, I plan to fit several multilevel models to the outcome variable of political trust. The levels are: individuals and countries, possibly with a longitudinal hierarchy. Moreover, for robustness tests factor analysis could form a scale (e.g. with the items: trust in neighbors/foreigners/people you first meet). Some salient interaction effects (e.g. economic performance x autocratic regimes) could be added.

However, in order to avoid issues of endogeneity (low generalized trust causing autocratic regimes) and validity (fear of answering honestly in respondents) I also plan to use instrumental variables with a Two Stage Least Squares Regression. These could be: corruption, log. of population density (s. Kim et al 2011)\nocite{kim2011}. Here, I am going to carefully examine possible confounders and relevance. 


\section{Paper Outline}


\begin{enumerate}
\item Introduction

\begin{itemize}
\item Research Gap
\item Short Summary
\item Outline
\end{itemize}

\item Literature Review
\begin{itemize}
\item Related Works in Trust Research
\item Related Works Autocratic Regimes
\end{itemize}

\item Theory
\begin{itemize}
\item Possible causal pathways (s. above)
\item Illustrative evidence (e.g. Stasi in Germany)
\end{itemize}

\item Definitions
\begin{itemize}
\item From Trust to Generalized Trust
\item Regime Types
\end{itemize}

\item Operationalization, Data and Method

\item Analysis
\begin{itemize}
\item Logistic Regressions
\end{itemize}

\item Robustness Test
\begin{itemize}
\item Instrumental Variables
\item OLS Regressions with Scale (or Index)
\end{itemize}
\end{enumerate}



%%%%%%%%%%%%%%%%%%%%%%%%%%%%%%%% 
% Abbildung                    %
%%%%%%%%%%%%%%%%%%%%%%%%%%%%%%%%

% Auch Abbildungen können wie gewohnt mit der figure Umgebung
% im Corporate Design der Universität Konstanz erstellt werden.



%%%%%%%%%%%%%%%%%%%%%%%%%%%%%%%% 
% Tabelle                      %
%%%%%%%%%%%%%%%%%%%%%%%%%%%%%%%%

% Um eine Tabelle im Corporate Design der Universität Konstanz
% zu erstellen wird das Paket tabu verwendet. Dadurch ergeben 
% sich auch kleine Unterschiede beim Erstellen von Tabellen. 
%
% Als Umgebung muss tabu anstatt tabular verwendet werden:
%
%    \begin{tabu}
%
%        ...
% 
%    \end{tabu}
%
% Die Spalten können direkt im Anschluss definiert werden.
% 
%    { X[coef, align, type] X[coef, align, type] ... }
%
%    - coef skaliert die Spalten, sollten es mehrere sein
%    - align ist entweder r, l, c oder j
%    - type ist entweder p (Standard), m oder b
%    - Vertikale Linien können mittels | zwischen den Spalten
%      gezeichnet werden. Dies sollte aus ästehtischen Gründen
%      jedoch wenn möglich vermieden werden.
%
% Danach können wie aus der tabular Umgebung gewohnt die
% Zeilen definiert werden. Die Spalte wird mit % gewechselt und
% ein Zeilenumbruch kann mit \\ eingeleitet werden.
%
% Möchte man eine horizontale Linie zeichnen, so können nach
% dem Corporate Design der Universität Konstanz entweder
%
%    \unitoprule
%
% für eine durchgezogene (dicke) Linie in seeblau oder
%
%    \unimidrule
%
% eine gestrichelte durchgezogene Linie in seeblau gezeichnet
% werden.
%
% Da die tabu Umgebung sehr mächtig ist, können auch weitere
% Varianten gezeichnet werden. Dazu sein an die Paketdokumentation
% verwiesen:
% 
%    ftp://ftp.fu-berlin.de/tex/CTAN/macros/latex/contrib/tabu/tabu.pdf
%
% Ein ausführliches Beispiel folgt gleich weiter unten.
%
% Die Tabelle sollte in einer table Umgebung eingebunden werden, damit
% sie im Tabellenverzeichnis erscheint. Außerdem kann noch eine Tabellen-
% überschrift hinzugefügt werden.

%%%%%%%%%%%%%%%%%%%%%%
% Schriftgröße       %
%%%%%%%%%%%%%%%%%%%%%%

% Da innerhalb einer Arbeit es meistens mehrere verschiedene Schriftgrößen
% gibt, steht hier das Makro
%
%     \selectfontsize
%
% zur Verfügung.
%
% Es kann jedoch auch die Latex internen Makros wie \Large, \small, o.ä. 
% verwendet werden.
%
% Dieses Makro besitzt ein unbedingt notwendiges Argument und ein optionales
% Feld, indem Key-Value Pairs übergeben werden können.
%
%     \selectfontsize[<Key Value Pairs>]{<Schriftgröße>}
%
% Das Argument hat dabei folgende Bedeutung:
% 
%   1. Argument:         Hier wird die neue Schriftgröße angegeben, die verwendet werden
%                        soll.
% Die weiteren Formatierungsoptionen werden alle innerhalb des optionalen Argumentes mittels
% Key-Value Pairs bestimmt.
%
% Dabei stehen folgende Optionen zur Verfügung:
%
%     baselineskip    Hier wird der baselineskip angegeben, welcher verwendet werden soll
%                     Mögliche Werte:
%                         0      Ist dieser 0, dann wird der baselinefaktor verwendet  
%                         sonst
%                    Standardwert: 0
%
%     baselinefaktor Hier wird der Faktor angegeben, der verwendet wird, um den neuen
%                    baselineskip zu berechnen.
%                    Dieser wird nur benutzt, falls der baselineskip 0 beträgt.
%                    
%                        baselineskip = baselinefaktor * #1
%
%                    Standardwert: 12/10
%
%                    Da hier keine Fließkommazahl in der Dezimalschreibweise angegeben werden
%                    kann, müssen diese als Brüche repräsentiert werden, wie z.b. 12/10 anstatt
%                    1.2.



%%%%%%%%%%%%%%%%%%%%%%%%%
% CD Element: Markieren %
%%%%%%%%%%%%%%%%%%%%%%%%%

% Um einen Text mit Hilfe des Markieren Elements des Corporate Design hervorzuheben,
% steht das Makro
%
%     \markieren
%
% zur Verfügung.
%
% Dieses Makro besitzt vier unbedingt notwendige Argumente und ein optionales
% Feld, indem weitere EIgenschaften festgelegt werden könnnen..
%
%     \markieren[Optionen per Key-Value Pair]{<Zeile 1>}{<Zeile 2>}{<Zeile 3>}{<Zeile 4>}
%
% Die Argumente haben dabei folgende Bedeutung:
%
%   1. - 4. Argument:    Hier werden nun die eigentlichen Zeilen übergeben.
%
%                        Wichtig dabei ist es, dass die Aufteilung der Zeilen manuell erfolgen muss durch
%                        die Argumente, da nur somit sichergestellt werden kann, dass bspw. Treppeneffekte
%                        nicht auftreten und somit der Benutzer alle Freiheiten bei der Aufteilung besitzt.
%
%                        Sollten nicht alle Zeilen verwendet werden, dann müssen die hinteren Brackets
%                        leer gelassen werden, wie beispielsweise bei der Headline
%
% Die wichtigen Formatierungsoptionen werden alle innerhalb des optionalen Argumentes mittels
% Key-Value Pairs bestimmt.
%
% Dabei stehen folgende Optionen zur Verfügung:
%
%   align                Hier kann angegeben werden, ob das komplette Objekt
%                        links- oder rechtsbündig angeordent werden soll.
%                      
%                        Der Standardwert ist "left" und somit linksbündig.
%
%                        Für eine rechtsbündige Anordnung muss hier der Wert "right" hinterlegt werden.
%
%   vertical             Hier wird angegeben, ob der Inhalt der Zeilen zentriert werden soll oder
%                        überall an der gleichen Baseline ausgerichtet werden soll.
%                       
%                        Dies kann mittels der Wörter "center" und "base" eingestellt werden.
%                        Dabei ist "center" als Standardwert festgelegt.
%
%                        Der Unterschied besteht darin, dass bei Zeilen die Buchstaben mit einer Tiefe
%                        enthalten, wie g, p oder q, anders zentriert werden als welche ohne Buchstaben
%                        mit einer Tiefe.
%
%                        Da dies ein wenig Geschmackssache ist, werden hier beide Varianten zur Verfügung
%                        gestellt, wobei "center" primär verwendet werden soll, und "base" eher wenn
%                        Buchstaben mit einer Tiefe in den Zeilen enthalten sind.



%%%%%%%%%%%%%%%%&%%%%%%%%%%%%%%
% CD Element: Unterstreichung %
%%%%%%%%%%%%%%%%%&%%%%%%%%%%%%%

% Um einen Text mit Hilfe des Unterstreichen Elements des Corporate Design hervorzuheben,
% steht das bereits bekannte Makro
%
%     \underline
%
% zur Verfügung, welches an die Anforderungen des Corporate Designs angepasst wurde.
%
% Dieses Makro besitzt ein notwendiges Argument
%
%     \underline{1. Argument}
%
% Das Argument hat folgende Bedeutung:
% 
%   1. Argument: Hier wird der zu unterstreichende Text hinterlegt.
%
% Wichtig ist noch zu wissen, dass auch Textbrüche ohne Probleme durchgeführt werden können.
%
% Zudem können weitere Formatierungen, wie bold oder italic innerhalb des Argumentes angewendet
% werden.
%
% Die Dicke der unterstrichenen Linie passt sich dabei der aktuell verwendeten Textgröße an.


%%%%%%%%%%%%%%%%%%%%%%
% CD Element: Merken %
%%%%%%%%%%%%%%%%%%%%%%

% Um einen Text mit Hilfe des Merken Elements des Corporate Design hervorzuheben,
% steht das Makro
%
%     \merken
%
% zur Verfügung.
%
% Dieses Makro besitzt drei unbedingt notwendige Argumente
%
%     \merken{1. Argument}{2. Argument}{3. Argument}
%
% Die Argumente haben dabei folgende Bedeutung:
% 
%   1. Argument: Hier wird die Breite des kompletten Objektes angegeben. Da das
%                Merken Objekt quadratisch ist, wird hier sowohl die Breite als auch
%                die Höhe angegeben.
%
%   2. Argument: Hier wird die Subline des Merken Elementes angegeben, die direkt unter der
%                Zeile mit dem X folgt (siehe auch Corporate Design Manual).
%
%   3. Argument: Hier wird der eigentliche Inhalt angegeben. Wichtig hierbei ist es,
%                dass dieser Inhalt an die untere Kante des Merken Elementes orientiert ist.
%                Somit entgegen der Subline (2. Argument), welche an die obere Kante abzüglich
%                der Zeile mit dem X orientiert ist.
%
% Hier folgt noch eine grafische Darstellung der Argumente:
%
%    |<------- 1. Argument ------->|    
%
%    -------------------------------    -
%    |                           X |    ^
%    | Subline (2. Argument)       |    |
%    |                             |    1
%    |                             |    .
%    |                             |    A
%    |                             |    r
%    |                             |    g
%    |                             |    u
%    |                             |    m
%    |                             |    e
%    |                             |    n
%    |                             |    t
%    |                             |    |
%    | Inhalt (3. Argument)        |    v
%    -------------------------------    -
%
% Wichtig ist noch zu wissen, dass die Linienstärke und die Größe des X in der rechten oberen Ecke an die
% Höhe / Breite des Merken Elements dynamisch angepasst ist. 

%%%%%%%%%%%%%%%%%%%%%%
% CD Element: Block  %
%%%%%%%%%%%%%%%%%%%%%%

% Mit dem Makro
%
%     \cdblock[Optionen per Key-Value Pair]{<Headline>}{<Spalte 1>}{<Spalte 2>}{<Spalte 3>}{<Spalte 4>}{<Spalte 5>}{<Spalte 6>}{<Spalte 7>}{<Spalte 8>}
%
% können Block-Elemente für z.B. wisschenschaftliche Inhalte erstellt werden.
%
% Dieses Makro besitzt 9 erforderliche Elemente, die bei jedem Aufruf angegeben werden müssen. Dabei 
% ist es natürlich mögliche Argumente leer zu lassen, falls man diese nicht benötigt. Dies hat jedoch
% keinen Einfluss auf die Anzahl an Spalten. Diese müssen separat im Optionenargument angegeben werden
% mittels des Schlüssels columnnum (siehe weiter unten).
%
% Die Argumente haben dabei folgende Bedeutung:
%
%   1. Argument: Inhalt der Headline
%   2. Argument: Inhalt der 1. Spalte
%   3. Argument: Inhalt der 2. Spalte
%   4. Argument: Inhalt der 3. Spalte
%   5. Argument: Inhalt der 4. Spalte
%   6. Argument: Inhalt der 5. Spalte
%   7. Argument: Inhalt der 6. Spalte
%   8. Argument: Inhalt der 7. Spalte
%   9. Argument: Inhalt der 8. Spalte
%
%
% Die wichtigen Formatierungsoptionen werden diesmal alle innerhalb des optionalen Argumentes mittels
% Key-Value Pairs bestimmt.
%
% Dabei stehen folgende Optionen zur Verfügung:
%
%     thick          Hier wird die Dicke der Linie bestimmt.
%                    Die Pfeile werden generell mit der doppelten Dicke gezeichnet!
%                    Standardwert: \boxlinewidth
%
%     color          Hier wird die Farbe der Linie angegeben
%                    Es sollten nur die folgenden Farben benutzt werden:
%                        seeblau100
%                        seeblau65
%                        seeblau35
%                        seeblau20
%                        black
%                        schwarz60
%                        schwarz40
%                        schwarz20
%                        schwarz10
%                    Standardwert: seeblau100
%
%     width          Hier wird die Breite des Blocks angegeben
%                    Standardwert: \paperwidth
%
%     columnnum      Hier werden die Anzahl an Spalten definiert
%                    Standardwert: 4
%
%     headlinesep    Hier wird der Abstand zwischen der Headline und den Spalten angegeben
%                    Standardwert: Aktuelle Schriftgröße
%
%     columnspace    Hier wird der Abstand zwischen den Spalten angegeben
%                    Standardwert: Doppelte Schriftgröße
%
%     block          Hier kann angegeben werden, ob man einen Rahmen um diesen Block haben möchte
%                    Mögliche Werte: true, false
%                    Standardwert: false
%
%     inner          Hier kann angegeben werden, ob zwischen den Spalten Trennlinien haben möchte
%                    Mögliche Werte:
%                        false    keine Trennlinien
%                        short    Trennlinien, die so lange sind, wie der längste Nachbar (entweder der
%                                 linke oder rechte Nachbar
%                        long     Trennlinien, die bis nach ganz unten gehen. Sie sind also so lang
%                                 wie die längste Spalte
%                    Standardwert: false
%
%     inner1,        Hier kann für jeden Zwischenraum der Spalte exakt angegeben werden, ob Trennlinien
%     inner2,        existieren sollen und falls ja, wie lang sie sein sollen. Diese Werte werden jedoch
%     inner3,        nur berücksichtigt, wenn inner=false ist. Ansonsten ist inner stärker.
%     inner4,        Mögliche Werte:
%     inner5,           false    keine Trennlinien
%     inner6,           short    Trennlinien, die so lange sind, wie der längste Nachbar (entweder der
%     inner7,                    linke oder rechte Nachbar
%                       long     Trennlinien, die bis nach ganz unten gehen. Sie sind also so lang
%                                wie die längste Spalte
%                    Standardwert: false
%
%     outerleft,     Hier kann angegeben werden, ob links (rechts) der ersten Spalte eine Trennlinie existieren soll.
%     outerright     Mögliche Werte
%                        false    keine Trennlinien
%                        short    Trennlinien, die so lange sind, wie der direkte Nachbar (bei outerleft die 1. Spalte
%                                 und bei outerright die letzte Spalte
%                        long     Trennlinien, die bis nach ganz unten gehen. Sie sind also so lang
%                                 wie die längste Spalte
%                        verylong Trennlinie geht von oben nach unten, sowie ein halber columnspace nach innen.
%                    Standardwert: false
%
%     outertop,      Hier kann angegeben werden, ob oberhalb (unterhalb) des Blocks eine Trennlinie, oder ein Pfeil existieren soll.
%     outerbottom    Mögliche Werte
%                        false    keine Trennlinien
%                        long     Trennlinien, die von links nach rechts geht
%                        verylong Trennlinie, die von links nach rechts geht, sowie ein halber columnspace nach oben (unten).
%                        arrow    Pfeil, der aus der Trenlinie verylong besteht und in der Mitte einen Pfeil nach oben (unten)
%                                 besitzt.
%                    Standardwert: false
%
%     arrowtop1left, Hier kann angegeben werden, ob zwei Spalten oberhalb des Blocks mittels eines Doppelpfeils verbunden werden
%     arrowtop1right sollen. Da es maximal 8 Spalten sind, können auch nur maximal 4 Paare bestimmt werden.
%     arrowtop2left  Ein Paar besteht somit aus einer linken und einer rechten Spalte.
%     arrowtop2right Mögliche Werte:
%     arrowtop3left      0   Keine Auswahl
%     arrowtop3right     1-8 Auswahl einer Spalte von 1 bis 8
%     arrowtop4left  Standardwert: 0
%     arrowtop4right Sollte der linke Wert nicht kleiner als der rechte Wert sein, so werden keine Pfeile gezeichnet. Das gleiche
%                    gilt für Werte, die außerhalb des Bereichs liegen.
% 
%
% Wichtig ist noch zu wissen, wie die Breite letztendlich berechnet wird:
% Da es auch links und rechts der Spalten Trennlinien oder Pfeile geben kann, ist links der 1. und rechts der
% letzten Spalte ebenfalls ein columnspace vorgesehen.
% Somit wird die Spaltenbreite wie folgt berechnet
%
%     blockcolumnwidth = (width - (columnspace * (columnnum + 1))) / columnnum
%


%%%%%%%%%%%%%%%%%%%%%%%%%%%%%%%%%%%%%%%%%%%%%%
% CD Element: Linie (mit optionalen Pfeilen) %
%%%%%%%%%%%%%%%%%%%%%%%%%%%%%%%%%%%%%%%%%%%%%%

% Mit dem Makro
%
%     \cdline[Optionen per Key-Value Pair]{<Länge der Linie>}
%
% kann eine Linie mit einer bestimmten Länge erstellt werden.
%
% Dieses Makro besitzt 1 erforderliches Element, welches bei jedem Aufruf mit angegeben
% werden muss.
%
% Das Argument hat dabei folgende Bedeutung.
%
%   1. Argument: Länge der erzeugten Linie
%
%
% Die wichtigen Formatierungsoptionen werden alle innerhalb des optionalen Argumentes mittels
% Key-Value Pairs bestimmt.
%
% Dabei stehen folgende Optionen zur Verfügung:
%
%     thick          Hier wird die Dicke der Linie bestimmt
%                    Standardwert: \boxlinewidth
%
%     mode           Hier wird angegeben, ob die Linie horizontal oder vertikal ausgerichtet werden soll
%                    Mögliche Werte
%                        horizontal
%                        vertical
%                    Standardwert: horizontal
%
%     color          Hier wird die Farbe der Linie angegeben
%                    Es sollten nur die folgenden Farben benutzt werden:
%                        seeblau100
%                        seeblau65
%                        seeblau35
%                        seeblau20
%                        black
%                        schwarz60
%                        schwarz40
%                        schwarz20
%                        schwarz10
%                    Standardwert: seeblau100
%
%     arrowleft      Hier wird angegeben, die Linie am linken (vertical: oberen) Ende mit einem Pfeil enden soll
%                    Mögliche Werte:
%                        true
%                        false
%                    Standardwert: false
%
%     arrowright     Hier wird angegeben, die Linie am rechten (vertical: unteren) Ende mit einem Pfeil enden soll
%                    Mögliche Werte:
%                        true
%                        false
%                    Standardwert: false


%%%%%%%%%%%%%%%%%%%%%%%%%%%%%%%%%%%%%%%%%%%%%%%%
% CD Element: Klammer (mit optionalen Pfeilen) %
%%%%%%%%%%%%%%%%%%%%%%%%%%%%%%%%%%%%%%%%%%%%%%%%

% Mit dem Makro
%
%     \cdbracket[Optionen per Key-Value Pair]{<Breite des Klammer>}{<Höhe des Klammer>}
%
% kann eine Klammer mit einer bestimmten Breite und Höhe gezeichnet werden.
%
% Dieses Makro besitzt 2 erforderliche Elemente, welche bei jedem Aufruf mit angegeben
% werden müssen.
%
% Die Argumente haben dabei folgende Bedeutung.
%
%   1. Argument: Breite der erzeugten Klammer
%   2. Argument: Höhe der erzeugten Klammer
%
%
% Die wichtigen Formatierungsoptionen werden alle innerhalb des optionalen Argumentes mittels
% Key-Value Pairs bestimmt.
%
% Dabei stehen folgende Optionen zur Verfügung:
%
%     thick          Hier wird die Dicke der Linien bestimmt
%                    Standardwert: \boxlinewidth
%
%     mode           Hier wird die Ausrichtung der Klammer angegeben
%                    Mögliche Werte
%                        left    linke Klammer
%                        top     obere Klammer
%                        right   rechte Klammer
%                        bottom  untere Klammer
%                    Standardwert: left
%
%     color          Hier wird die Farbe der Linie angegeben
%                    Es sollten nur die folgenden Farben benutzt werden:
%                        seeblau100
%                        seeblau65
%                        seeblau35
%                        seeblau20
%                        black
%                        schwarz60
%                        schwarz40
%                        schwarz20
%                        schwarz10
%                    Standardwert: seeblau100
%
%     arrowleft      Hier wird angegeben, ob die Klammer am linken (oberen) Ende mit einem Pfeil enden soll
%                    Mögliche Werte:
%      

\cleardoublepage




%In Material und Methoden (kann bei einer ausschließlichen Literaturarbeit entfallen) werden je nach Thematik u.?a. die Lage der Versuchsflächen, die Untersuchungsbestände, die Versuchsanordnung, die Aufnahmemethodik und die statistische Auswertung exakt beschrieben. Die Angaben sollten so ausführlich sein, dass ein Unbeteiligter die Untersuchungen noch einmal durchführen kann. 




\chapter{Material and methods}

\section{Feeding Trial}

The ideal approach for a animal feeding trial in most cases should be done on a farm in a real-live situation. While not every feeding trial initially includes objectives to test for resurce availability, ressurce management or technical feasibility, those objectives will become an issue in the further investigation and need to be adressed sooner or later. Farmers can serve as a forum to discuss practical problems and can provide
insight for appropriate settings and technological adaptation. In optimal conditions, the researching facility would have their own real-live sized ponds with experianced staff. With such a setting every species under various conditions with multible replicates can be tested.Farmers can usually only adress well established species or even breeds  and only accept one treatment and few replicates. The financial aspect is the major issue in feeding trials because a total loss of the  population must be accounted for and therefore a lot of farmers are careful with experimental setups and usually demand compensatory payment in case of partially/total loss. \\

On this regard, it usually makes sense to go with a small-scale experimental design for preliminary investigation. In such a design a variaty of species, water conditions and feed formulations can be adressed with enough replicates to make statistic analysis work. This design can be very demanding as there is usually a lack in well established technology and most parameters need to be controlled manually. The manual demand opens up the possibility for human errors, skilled and optimally 1-2 operators are advised to carry out such experiments to decrease interpersonal errors/failures. The major advantage of this design is, that in a short time the "best shot" can be drafted out to go in a real-live farm sized setup although the adaptation from a manual approach to a large sized technical approach ca be very hard.\\

For this feeding trial, a small scale design was choosen as there was a lack of knowledge in feed formulation (will it arrive in the animal), mode of operation of the ingredient (TG Probiotic, is it active and will it stay in the animal or interact) and mode of functioning (does the ingredient have an effect on the animal and if so, is the effect desirable).  

\section{Experimental Design}

The experiment was carried out at the aquaculture lab of FiBL. The facility consists of 30 aquaria, 3 rows of 10 aquaria each. Each aquarium is made of glass and can hold up to 80 l water, they are sealed of with white blinds so the populations cant see or interact with each other. The watersupply comes from a well with a WASSERHÄRTE, saturated oxigen levels and a pH of 7,9. The temperature is mostly dependant on the season, between 01.10 and 01.12 the water temperature fluctuated between 10-12  C depending on heavy rain falls and chilly weather. The water intake of each aquarium can be controlled  seperately, the usual water intake was  1 l/min but heavily polluted aquaria where upregulated for several hours to keep nitrogen loads and nitrogen induced stress low. Additionally each aquarium had a individual air intake with a air stone (terratec AS 40), powered by a centralized airpump (LUFTPUMPE) to keep oxigen saturation above 70  (60 \% be dangerous). The pollution with feces was kept low with external waterfilters that where cleaned on a weekly basis, feces traps (half a plant pot in a corner) helped a lot to siphone out the feces on a daily basis. Each aquarium was also topped with a lid, as rainbow trout tend to flee with a jump through the water surface to escape aggressive behaviour.
Two different parameters are to be tested. The influence of the probiotic (on X, Y, Z hypothesen) and the influence of Nori algae in combination with the probiotic (Hypothese 1,2,3). To test for the two different  parameters, 4 treatments with 7 replicates (total of 28 aquaria, 2 empty) are set up.
The treatments are labled as follows: C- indicates the control feed without
probiotic supplement, C+ indicates the control feed with probiotic supplement,
A- indicates a feed formulation with 10 \% nori [a]lgae but without probiotic 
supplement and finally A+ which contains both 10 \% nori algae and the probiotic
supplement. 

Skizze

Beschriftung: SUPER HEFTIG SKIZZE

\section{Rainbow Trout}

Initial stocking of fish was 20 animals at 15g +-STD in each aquarium with
a stocking density of  8 kg/m. In total 600 rainbow trout (oncorynchus mykiss) where ordered from
the rainbow trout farmer PIMPELHUBER. The animals where taken shortly bevore
the red mouth disease (RMD) vacaccination in the nursary took place, this was mandatory because a 
small part of the fish will go into a infection experiment at the university Bern
where they use RMD as a infection model. The fish where transferred in 3 Bags
filled with technical oxigen to withstand the 3 hour distance to FiBL. The fish 
where transferred into a bigger pond with 1 m3 and left there for the two week
acclimatisation. They where fed FUTTERNAME SKRETTING provided by the farmer.
 After acclimatiosation, 30 fish of approximately same size where
weighted and transferred in the final aquaria. The initial stocking is furthermore
the start of the experiment, from that day they get their associated experimental feed.

\section{Feed formulation}

The optimal solution for feed formulation would be to order from a feedstuff mill
as they have the knowledge, routine and equipment (extruder) for an optimal extruded
fish feed. At a daily feeding rate of 2 \% on a total of 6 week trial (Excluding the feed needed for the following
infection experiment), each treatment needs roughly 3 kg of feed. 
The smallest batch from a feedstuff mill is usually 0,3-1 metric
ton which exceeds not only the price budged but also storing capacity and leaves huge
quantitys of unneeded leftovers behind. 
Facing those problems the feed was manually produced with a meat grinder at FiBL.
To get the mechanical properties of a extruded feed which is precisely adapted for 
aquatic feeding, a standard organic rainbow trout feed (Natura Trout 2 mm)from the feedstuff mill HOKOVIT
was used. HOKOVIT is the biggest aquaculture feed producer in switzerland (although
local aquaculture currently represents only a small market in switzerland) and has
a long history in working together with the FiBL. 
The feed pellets where ground with a HAMMERSCHLAGMÜHLE 3000 to a fine powder
and stored in a cooling room for further usage. 
Nori Algae (FIRMA, ungerösted, 25g Packung) where ground with a SCHNEIDMÜHLE 3000, and stored
in a cooling room for further usage.
The probiotic supplement was obtained by the StartUp twentygreen as a fine powder ready
to use in a dark, sealed bottle to avoid UV breakdown.
The last ingredient, titan dioxid-2 (Sigma Aldrich, 100g bottle PA) was ordered as
a fine powder as well, ready for use.
All ingredients where mixed dry according to their treatments. Treatments with
probiotic supplement contained 0,1 \% of the probiotic (cell count unknown), Algae Treatments contained
10 \% nori algae powder and all treatments where added 1 \% of titan dioxid as a digestion
marker. After dry mixing, destilled water (NICHT DESTILLIERT) was added at a dilution of
1:2 (1 part water, 2 parts feed) and knead by hand until a nonsticking dough emerges.
The feed dough was run through twice in the meat grinder with a 2 mm grid for better distribution
of the ingredients, layed out on tin foil for drying in a drying oven at 50 C over night and
stored in cooling room for final feeding.

\section{Maintainance}

Daily routine includes visual checking for fish welfare. In a previous experiment
with only 5 individual rainbow trout per aquaria, a very agressive behaviour 
could be observed where usually one dominant animal suppresses, attacks and stops
other animals from feeding. The attacks of the dominant animal in some cases ended
with victims loosing their dynamic equilibrium, open flesh wounds and heavy
fin lesions. In this experiment a protocol for detecting moribund animals was used
to remove critical animals ethically correct. 
Daily siphoning guaranteed low particle pollution and low nitrogen levels. Once
a week water analytic was carried out, testing for ammonium and nitrate levels in
the water. The temperature, pH, oxigen level and saturation was monitored closely
on a regulare basis.
Every monday during the whole experiment, all fish populations where cought with
a net and wet weight to adjust the amount of feed they need. Prior to the weighting
procedure no feed for 24 hours was given to ensure sober weight. 





\cleardoublepage





%%%%%%%%%%%%%%%%%%%%%%%%%%%%%%%% 
% Literaturverzeichnis         %
%%%%%%%%%%%%%%%%%%%%%%%%%%%%%%%%

% Zum Schluss kann noch das Literaturvzerzeichnis hinzugefügt
% werden.
%
% Damit es ebenfalls im Inhaltsverzeichnis gefunden werden kann,
% sollte das Literaturverzeichnis mit dem Makro
%
%    \addcontentsline{toc}{chapter}{Literaturverzeichnis}
%
% hinzugefügt werden.


\begin{thebibliography}{[MB91a]}
\addcontentsline{toc}{chapter}{Literaturverzeichnis}

\normalsize
\sffamily

\setlength{\itemsep}{6pt}

\bibitem[Cd15]{bib:corporateDesign}
%Universität Konstanz: Corporate Design Manual.
%Universtät Konstanz, (2015)

\end{thebibliography}


%%%% NEU - mit BibTex

\bibliographystyle{alphadin}
\bibliography{references} 

%%%%%%%%%%%%%%%%%%%%%%%%%%%%%%%%%%%%%%%%%%%%%%%%%%%%%%%%%
% Ende vom Dokument                                     %
%%%%%%%%%%%%%%%%%%%%%%%%%%%%%%%%%%%%%%%%%%%%%%%%%%%%%%%%%
\end{document}