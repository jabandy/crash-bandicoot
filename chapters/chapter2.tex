


%In Material und Methoden (kann bei einer ausschließlichen Literaturarbeit entfallen) werden je nach Thematik u.?a. die Lage der Versuchsflächen, die Untersuchungsbestände, die Versuchsanordnung, die Aufnahmemethodik und die statistische Auswertung exakt beschrieben. Die Angaben sollten so ausführlich sein, dass ein Unbeteiligter die Untersuchungen noch einmal durchführen kann. 




\chapter{Material and methods}

\section{Feeding Trial}

The ideal approach for a animal feeding trial in most cases should be done on a farm in a real-live situation. While not every feeding trial initially includes objectives to test for resurce availability, ressurce management or technical feasibility, those objectives will become an issue in the further investigation and need to be adressed sooner or later. Farmers can serve as a forum to discuss practical problems and can provide
insight for appropriate settings and technological adaptation. In optimal conditions, the researching facility would have their own real-live sized ponds with experianced staff. With such a setting every species under various conditions with multible replicates can be tested.Farmers can usually only adress well established species or even breeds  and only accept one treatment and few replicates. The financial aspect is the major issue in feeding trials because a total loss of the  population must be accounted for and therefore a lot of farmers are careful with experimental setups and usually demand compensatory payment in case of partially/total loss. \\

On this regard, it usually makes sense to go with a small-scale experimental design for preliminary investigation. In such a design a variaty of species, water conditions and feed formulations can be adressed with enough replicates to make statistic analysis work. This design can be very demanding as there is usually a lack in well established technology and most parameters need to be controlled manually. The manual demand opens up the possibility for human errors, skilled and optimally 1-2 operators are advised to carry out such experiments to decrease interpersonal errors/failures. The major advantage of this design is, that in a short time the "best shot" can be drafted out to go in a real-live farm sized setup although the adaptation from a manual approach to a large sized technical approach ca be very hard.\\

For this feeding trial, a small scale design was choosen as there was a lack of knowledge in feed formulation (will it arrive in the animal), mode of operation of the ingredient (TG Probiotic, is it active and will it stay in the animal or interact) and mode of functioning (does the ingredient have an effect on the animal and if so, is the effect desirable).  

\section{Experimental Design}

The experiment was carried out at the aquaculture lab of FiBL. The facility consists of 30 aquaria, 3 rows of 10 aquaria each. Each aquarium is made of glass and can hold up to 80 l water, they are sealed of with white blinds so the populations cant see or interact with each other. The watersupply comes from a well with a WASSERHÄRTE, saturated oxigen levels and a pH of 7,9. The temperature is mostly dependant on the season, between 01.10 and 01.12 the water temperature fluctuated between 10-12  C depending on heavy rain falls and chilly weather. The water intake of each aquarium can be controlled  seperately, the usual water intake was  1 l/min but heavily polluted aquaria where upregulated for several hours to keep nitrogen loads and nitrogen induced stress low. Additionally each aquarium had a individual air intake with a air stone (terratec AS 40), powered by a centralized airpump (LUFTPUMPE) to keep oxigen saturation above 70  (60 \% be dangerous). The pollution with feces was kept low with external waterfilters that where cleaned on a weekly basis, feces traps (half a plant pot in a corner) helped a lot to siphone out the feces on a daily basis. Each aquarium was also topped with a lid, as rainbow trout tend to flee with a jump through the water surface to escape aggressive behaviour.
Two different parameters are to be tested. The influence of the probiotic (on X, Y, Z hypothesen) and the influence of Nori algae in combination with the probiotic (Hypothese 1,2,3). To test for the two different  parameters, 4 treatments with 7 replicates (total of 28 aquaria, 2 empty) are set up.
The treatments are labled as follows: C- indicates the control feed without
probiotic supplement, C+ indicates the control feed with probiotic supplement,
A- indicates a feed formulation with 10 \% nori [a]lgae but without probiotic 
supplement and finally A+ which contains both 10 \% nori algae and the probiotic
supplement. 

Skizze

Beschriftung: SUPER HEFTIG SKIZZE

\section{Rainbow Trout}

Initial stocking of fish was 20 animals at 15g +-STD in each aquarium with
a stocking density of  8 kg/m. In total 600 rainbow trout (oncorynchus mykiss) where ordered from
the rainbow trout farmer PIMPELHUBER. The animals where taken shortly bevore
the red mouth disease (RMD) vacaccination in the nursary took place, this was mandatory because a 
small part of the fish will go into a infection experiment at the university Bern
where they use RMD as a infection model. The fish where transferred in 3 Bags
filled with technical oxigen to withstand the 3 hour distance to FiBL. The fish 
where transferred into a bigger pond with 1 m3 and left there for the two week
acclimatisation. They where fed FUTTERNAME SKRETTING provided by the farmer.
 After acclimatiosation, 30 fish of approximately same size where
weighted and transferred in the final aquaria. The initial stocking is furthermore
the start of the experiment, from that day they get their associated experimental feed.

\section{Feed formulation}

The optimal solution for feed formulation would be to order from a feedstuff mill
as they have the knowledge, routine and equipment (extruder) for an optimal extruded
fish feed. At a daily feeding rate of 2 \% on a total of 6 week trial (Excluding the feed needed for the following
infection experiment), each treatment needs roughly 3 kg of feed. 
The smallest batch from a feedstuff mill is usually 0,3-1 metric
ton which exceeds not only the price budged but also storing capacity and leaves huge
quantitys of unneeded leftovers behind. 
Facing those problems the feed was manually produced with a meat grinder at FiBL.
To get the mechanical properties of a extruded feed which is precisely adapted for 
aquatic feeding, a standard organic rainbow trout feed (Natura Trout 2 mm)from the feedstuff mill HOKOVIT
was used. HOKOVIT is the biggest aquaculture feed producer in switzerland (although
local aquaculture currently represents only a small market in switzerland) and has
a long history in working together with the FiBL. 
The feed pellets where ground with a HAMMERSCHLAGMÜHLE 3000 to a fine powder
and stored in a cooling room for further usage. 
Nori Algae (FIRMA, ungerösted, 25g Packung) where ground with a SCHNEIDMÜHLE 3000, and stored
in a cooling room for further usage.
The probiotic supplement was obtained by the StartUp twentygreen as a fine powder ready
to use in a dark, sealed bottle to avoid UV breakdown.
The last ingredient, titan dioxid-2 (Sigma Aldrich, 100g bottle PA) was ordered as
a fine powder as well, ready for use.
All ingredients where mixed dry according to their treatments. Treatments with
probiotic supplement contained 0,1 \% of the probiotic (cell count unknown), Algae Treatments contained
10 \% nori algae powder and all treatments where added 1 \% of titan dioxid as a digestion
marker. After dry mixing, destilled water (NICHT DESTILLIERT) was added at a dilution of
1:2 (1 part water, 2 parts feed) and knead by hand until a nonsticking dough emerges.
The feed dough was run through twice in the meat grinder with a 2 mm grid for better distribution
of the ingredients, layed out on tin foil for drying in a drying oven at 50 C over night and
stored in cooling room for final feeding.

\section{Maintainance}

Daily routine includes visual checking for fish welfare. In a previous experiment
with only 5 individual rainbow trout per aquaria, a very agressive behaviour 
could be observed where usually one dominant animal suppresses, attacks and stops
other animals from feeding. The attacks of the dominant animal in some cases ended
with victims loosing their dynamic equilibrium, open flesh wounds and heavy
fin lesions. In this experiment a protocol for detecting moribund animals was used
to remove critical animals ethically correct. 
Daily siphoning guaranteed low particle pollution and low nitrogen levels. Once
a week water analytic was carried out, testing for ammonium and nitrate levels in
the water. The temperature, pH, oxigen level and saturation was monitored closely
on a regulare basis.
Every monday during the whole experiment, all fish populations where cought with
a net and wet weight to adjust the amount of feed they need. Prior to the weighting
procedure no feed for 24 hours was given to ensure sober weight. 




